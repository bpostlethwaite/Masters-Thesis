%% ------------------------------------------------------------------------ %%
\documentclass[draft]{article}

\title{Seismic parameter estimation and the Canadian crust}
\author{Ben Postlethwaite}

%% ------------------------------------------------------------------------ %%
\begin{document}


%% ------------------------------------------------------------------------ %%
\begin{abstract}
   It has been suggested that processes driving crustal formation in the Archean and Proterozoic were dissimilar and produced crusts with unique bulk properties and average thicknesses. Existing models based on fractionating mantle composition or evolving mantle convection require accurate estimates of the geological and geophysical properties of crustal provinces to better understand the details of early continental formation (R. Durrheim, W. Mooney, 1991). Fifteen years of publicly accessible teleseismic data from all available Canadian seismic stations are binned in horizontal slowness and deconvolved into receiver functions. We apply a new stacking method (Bostock and Kumar, 2010) to retrieve estimates of bulk crustal velocities (Vp, Vs) and thickness H from these data under the assumption of 1-D structure. Bootstrap error analysis is performed for each station dataset and results are compared to the results produced from alternate methods (Zhu and Kanamori, 2000). Cross-analyzing these results with the mineral and rock seismic property database of Christensen (1996) will afford improved constraints on bulk geological composition of the Canadian landmass. These results will be used to evaluate competing models of early crustal formation.
\end{abstract}

%% ------------------------------------------------------------------------ %%

%% ------------------------------------------------------------------------ %%
%% -----------------------------------------------%%
\section{Introduction}
(Article text here)
\subsection{Overview}
(Overview here)
\subsection{Geological Background}
(Geological background here)

%% -----------------------------------------------%%
\section{Data and method}
   Seismic data utilized in this study comes from 343 stations across Canada from all available networks. Of these stations 146 were used for analysis. All stations are broadband with data being used between the years 2000 and 2012. Receiver function analysis has been widely employed over many years to investigate the earths structure.
   Two derivations have been used in this study, one well tested stacking approach (Zhu and Kanamori, 2000) and the other a recent method that has not been employed at scale (Bostock, 2010). Both methods rely on the fact that the incoming S-wave contains energy from the direct arrival as well as reflected phases resulting from sharp velocity contrasts. Deconvolution of the P-wave energy arrival, used as an estimate for the source function, from the S-wave signal produces a Green's function with energy peaks at the times of the main and reflected arrivals. For better separation of P and S wave energy all components are rotated into radial and transverse dimensions and transformed into P, SV and SH components with a wave field decomposition transfer matrix (Bostock, 1998).
   Deconvolution is performed by minimizing the general cross validation function $GCV(\delta)$ where $\delta$ is used the regularizer (Bostock, 1998; Golub et al., 1979). This method is an $L_2$ frequency domain deconvolution which performs quickly and makes no assumptions on the noise in the data. All deconvolved signals are filtered between 0.04hz and 3.0hz and are stacked with weights $w_1=0.5$, $w_1=0.3$, $w_1=0.2$ for the Ps, PpPs and PpSs phases respectively.

   <Error Calculations>

\subsection{Vp methods}
   The stacking approach outlined above provides estimates for the parameters $R=\frac{V_P}{V_S}$ as well as crustal thickness $H$ - which gives the depth of the Moho. A method which estimates $V_P$ and $V_S$ separately (Bostock, 2010) is also employed for select stations. This method makes use of the fact the dependence on $H$ in the travel time equations can be removed if we divide the reflected phases by the main arrival.
$$ t_{Pps}(p_i) = \frac{ \sqrt{R^2-p_i^2V_P^2} + \sqrt{1-p_i^2V_P^2} } {\sqrt{R^2-p_i^2V_P^2} + \sqrt{1-p_i^2V_P^2} } t_{Ps}(p_i) $$
$$ t_{Pss}(p_i) = \frac{2\sqrt{R^2-p_i^2V_P^2}} {\sqrt{R^2-p_i^2V_P^2} + \sqrt{1-p_i^2V_P^2} } t_{Ps}(p_i) $$
This has the advantage that no assumptions on $V_P$ are necessary to perform the gridsearch and stack. A similar stacking method as the Zhu and Kanamori approach may be employed for $t_{Pps}$ and $t_{Pss}$ as long as an estimate for $t_{Ps}$ exists. With estimates for $R$ and $V_P$ a simple line search along $H$ can be made using all three undivided travel time equations.

Several methods for choosing $t_{Ps}$ are available. Direct arrival $t_{Ps}$ should contain the largest fraction of energy out of the other phases and the moveout, depending on Moho depth, should be in the order of 3 to 6 seconds. Given this it is trivial to define a window and select the time corripsonding to maximum amplitude. Another approach is to use max amplitude estimates as data in a non-linear optimization to find the $R$, $V_P$, and $H$ which minimizes the residual between $t_{Ps}$ and the data. The travel time equations are twice-differentiable so the quadratically convergent Newton's method may be employed. This approach has the advantage that noise leading to poor maximum amplitude picks are effectively collapsed onto the curve corrisponding to the travel time function. A third approach which offers picks along a travel time curve but better stability than the non-linear method is to perform a gridsearch with all three travel time functions, a full Kanamori stack. With the best estimates for $R$ and $H$ the $t_{Ps}$ function can be found and used as input into the Bostock method. The trade-off is that the requirement for an initial $V_P$ estimate for the initial stack introduces secondary $V_P$ dependence into the system.


%% -----------------------------------------------%%
\section{Results}
<Results go here>

%% -----------------------------------------------%%
\section{Discussion}
\subsection{Canada}
<Discussion on larger - Canada wide - regions>
\subsection{Slave Province}
<Discussion on Slave Province>

%% -----------------------------------------------%%
\section{Conclusions}
<Conclusions here>

%% ------------------------------------------------------------------------ %%


% AGU does not want a .bib or a .bbl file. Please copy in the contents of your .bbl file here.

\begin{thebibliography}{}


\end{thebibliography}

%Reference citation examples:

%...as shown by \textit{Kilby} [2008].
%...as shown by {\textit  {Lewin}} [1976], {\textit  {Carson}} [1986], {\textit  {Bartholdy and Billi}} [2002], and {\textit  {Rinaldi}} [2003].
%...has been shown [\textit{Kilby et al.}, 2008].
%...has been shown [{\textit  {Lewin}}, 1976; {\textit  {Carson}}, 1986; {\textit  {Bartholdy and Billi}}, 2002; {\textit  {Rinaldi}}, 2003].
%...has been shown [e.g., {\textit  {Lewin}}, 1976; {\textit  {Carson}}, 1986; {\textit  {Bartholdy and Billi}}, 2002; {\textit  {Rinaldi}}, 2003].

%...as shown by \citet{jskilby}.
%...as shown by \citet{lewin76}, \citet{carson86}, \citet{bartoldy02}, and \citet{rinaldi03}.
%...has been shown \citep{jskilbye}.
%...has been shown \citep{lewin76,carson86,bartoldy02,rinaldi03}.
%...has been shown \citep [e.g.,][]{lewin76,carson86,bartoldy02,rinaldi03}.
%
% Please use ONLY \citet and \citep for reference citations.
% DO NOT use other cite commands (e.g., \cite, \citeyear, \nocite, \citealp, etc.).

%% ------------------------------------------------------------------------ %%


\end{document}

%% ------------------------------------------------------------------------ %%


More Information and Advice:


%
%  SECTION HEADS
%
%% ------------------------------------------------------------------------ %%

% Capitalize the first letter of each word (except for
% prepositions, conjunctions, and articles that are
% three or fewer letters).

% AGU follows standard outline style; therefore, there cannot be a section 1 without
% a section 2, or a section 2.3.1 without a section 2.3.2.
% Please make sure your section numbers are balanced.
% ---------------
% Level 1 head
%
% Use the \section{} command to identify level 1 heads;
% type the appropriate head wording between the curly
% brackets, as shown below.
%
%An example:
%\section{Level 1 Head: Introduction}
%
% ---------------
% Level 2 head
%
% Use the \subsection{} command to identify level 2 heads.
%An example:
%\subsection{Level 2 Head}
%
% ---------------
% Level 3 head
%
% Use the \subsubsection{} command to identify level 3 heads
%An example:
%\subsubsection{Level 3 Head}
%
%---------------
% Level 4 head
%
% Use the \subsubsubsection{} command to identify level 3 heads
% An example:
%\subsubsubsection{Level 4 Head} An example.
%
%% ------------------------------------------------------------------------ %%
%
%  IN-TEXT LISTS
%
%% ------------------------------------------------------------------------ %%
%
% Do not use bulleted lists; enumerated lists are okay.
% \begin{enumerate}
% \item
% \item
% \item
% \end{enumerate}
%
%% ------------------------------------------------------------------------ %%
%
%  EQUATIONS
%
%% ------------------------------------------------------------------------ %%

% Single-line equations are centered.
% Equation arrays will appear left-aligned.

Math coded inside display math mode \[ ...\]
 will not be numbered, e.g.,:
 \[ x^2=y^2 + z^2\]

 Math coded inside \begin{equation} and \end{equation} will
 be automatically numbered, e.g.,:
 \begin{equation}
 x^2=y^2 + z^2
 \end{equation}

% IF YOU HAVE MULTI-LINE EQUATIONS, PLEASE
% BREAK THE EQUATIONS INTO TWO OR MORE LINES
% OF SINGLE COLUMN WIDTH (20 pc, 8.3 cm)
% using double backslashes (\\).

% To create multiline equations, use the
% \begin{eqnarray} and \end{eqnarray} environment
% as demonstrated below.
\begin{eqnarray}
  x_{1} & = & (x - x_{0}) \cos \Theta \nonumber \\
        && + (y - y_{0}) \sin \Theta  \nonumber \\
  y_{1} & = & -(x - x_{0}) \sin \Theta \nonumber \\
        && + (y - y_{0}) \cos \Theta.
\end{eqnarray}

%If you don't want an equation number, use the star form:
%\begin{eqnarray*}...\end{eqnarray*}

% Break each line at a sign of operation
% (+, -, etc.) if possible, with the sign of operation
% on the new line.

% Indent second and subsequent lines to align with
% the first character following the equal sign on the
% first line.

% Use an \hspace{} command to insert horizontal space
% into your equation if necessary. Place an appropriate
% unit of measure between the curly braces, e.g.
% \hspace{1in}; you may have to experiment to achieve
% the correct amount of space.


%% ------------------------------------------------------------------------ %%
%
%  EQUATION NUMBERING: COUNTER
%
%% ------------------------------------------------------------------------ %%

% You may change equation numbering by resetting
% the equation counter or by explicitly numbering
% an equation.

% To explicitly number an equation, type \eqnum{}
% (with the desired number between the brackets)
% after the \begin{equation} or \begin{eqnarray}
% command.  The \eqnum{} command will affect only
% the equation it appears with; LaTeX will number
% any equations appearing later in the manuscript
% according to the equation counter.
%

% If you have a multiline equation that needs only
% one equation number, use a \nonumber command in
% front of the double backslashes (\\) as shown in
% the multiline equation above.

%% ------------------------------------------------------------------------ %%
%
%  LANDSCAPE/SIDEWAYS FIGURE AND TABLE EXAMPLES
%
%% ------------------------------------------------------------------------ %%
%
% For figures, add \usepackage{lscape} to the file and the landscape.sty style file
% to the paper folder.
%
% \begin{figure*}[p]
% \begin{landscapefigure*}
% Illustration here.
% \caption{caption here}
% \end{landscapefigure*}
% \end{figure*}
%
% For tables, add \usepackage{rotating} to the paper and add the rotating.sty file to the folder.
%
% AGU prefers the use of {sidewaystable} over {landscapetable} as it causes fewer problems.
%
% \begin{sidewaystable}
% \caption{}
% \begin{tabular}
% Table layout here.
% \end{tabular}
% \end{sidewaystable}
%
%

